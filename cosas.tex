\documentclass{article}
\usepackage{graphicx} % Required for inserting images

\title{Cosas para hacer}
\author{GLyC}
\date{\today}

\begin{document}

\maketitle


%%%%%%%%%%%%%%%%%%%%%%%%%%%%%%%%%%%%%%%
\section{General}

\begin{itemize}
    \item Cambiar el modelo del paper para que acepte cuando imprime un estado aceptador
\end{itemize}




%%%%%%%%%%%%%%%%%%%%%%%%%%%%%%%%%%%%%%%
\section{Normalización}

\begin{itemize}
    \item Generalizar el proceso de normalización a alfabetos más grandes $\rightarrow$ hacer una especie de torneo como el de dos funcionaría. Hacer una suerte de llave.
\end{itemize}




%%%%%%%%%%%%%%%%%%%%%%%%%%%%%%%%%%%%%%%
\section{Primitivas}

\begin{itemize}
    \item Implementar operadores booleanos $\rightarrow$ fácil con normalización y sin CoT
\end{itemize}


%%%%%%%%%%%%%%%%%%%%%%%%%%%%%%%%%%%%%%%
\section{Alfabeto}

\begin{itemize}
    \item Repensar todo usando un alfabeto $\Sigma$ sin forzar a que sea $\{0,1\}$. Debería valer todo lo mismo. Esta bueno que podemos agregar símbolos de parada. Nos permite repensar cuando deja de hacer CoT. (la idea es que sea cuando imprime el símbolo de parada (usar uno para no y uno para si) y que lo haga antes de $n_{max}$)
    \item Ambicioso: probar que podemos simular cualquier alfabeto con $\{0,1\}$. No vi este resultado en ningún lado.
\end{itemize}



%%%%%%%%%%%%%%%%%%%%%%%%%%%%%%%%%%%%%%%
\section{CoT}

\begin{itemize}
    \item Correr dos transformers a la vez pero con CoT. Fácil si tenemos un alfabeto más rico que $\{0,1\}$
\end{itemize}



%%%%%%%%%%%%%%%%%%%%%%%%%%%%%%%%%%%%%%%
\section{Aleatoriedad}
\begin{itemize}
    \item Definir el modelo sampleando
    \item Podemos copiarnos la def y hay que definir las probas recursivamente. Hacer de cuenta que tenes un sampleador que le tiras la distribución y te responde. Pensar cómo funciona eso y cuantos bits aleatorios necesita.
    \item Empezar a pensar en clases --> Notar cosas como cerrado por op booleanas y cosas de voto mayoritario
    \item Pensar cómo funcionaría lo del sampleo
    \item Escribir que podemos determinizar uno que samplea (usas normalización $\infty$)
\end{itemize}




\end{document}
