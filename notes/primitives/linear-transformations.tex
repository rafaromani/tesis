\section*{Linear transformations}

In this section we will see how to apply a linar transformation to one of the vectors and save the output in another more to the right, in other words: $h_j^{l+c} = M h_i^l$ for some $i < j$ previously chosen.

Note that after this procedure all the vector will have their values corrupted with the exception of $h_j$.

\bigskip

First we will make $h_j = 0$ at the coordinates we are interested. Then with an ATTN layer we will add $M h_i$.

Thanks to the previous sections we can assume that $h_i$ is the only vector which has $(1,0)$ in the lasts coordinates and that the rests end in $(0,1)$. These numbers would have been set by the encoding and preserved trough the previous layers.

Assuming this will allow us to construct $W_K$ and $W_Q$ such that:

\[
\langle q_{j}, k_{i'} \rangle = \begin{cases}
    -\infty &\text{ if } i' \neq i \\
    0  &\text{ if } i' = i 
\end{cases}
\]

Note that the $W_Q$ and $W_K$ that we are looking for are the following:

\begin{align*}
    & W_Q = \left(\begin{matrix}
    &0      &\hdots &0      &0          \\
    &\vdots &\ddots &\vdots &\vdots     \\
    &0      &\hdots &0      &0          \\
    &0      &\hdots &0      &-\infty    \\
\end{matrix}\right)
    &W_K = \left(\begin{matrix}
    &0      &\hdots &0      &0      &0      \\
    &\vdots &\ddots &\vdots &\vdots &\vdots \\
    &0      &\hdots &0      &0      &0      \\
    &0      &\hdots &0      &0      &1      \\
\end{matrix}\right)
\end{align*}


since


\begin{align*}
    &q_{j} = W_Q \; h_j = \left(\begin{matrix}
        0 \\
        \vdots \\
        0 \\
        -\infty
    \end{matrix}\right)
    &k_{i'} = W_K \; h_{i'} = \left(\begin{matrix}
        0 \\
        \vdots \\
        0 \\
        (h_{i'})_{d} \\
    \end{matrix}\right)
\end{align*}

so $\langle q_{j}, k_{i'} \rangle = -\infty *(h_{i'})_{d}$

\bigskip

Taking $W_O = M$ y $W_V = id$ we get what we were looking for. This happens because the fact that $e^{-\infty} = 0$ makes that $softmax(-\infty, \dots, -\infty, 0) = (0, \dots, 0, 1)$, which makes $(s_j)_{i'} = \mathbb{I}(i' = i)$ so finally:

\[
  W_O \sum_{i'=0}^{n} (s_j)_{i'} v_{i'} = 
  W_O \sum_{i'=0}^{j} (s_j)_{i'} v_{i'} = 
  W_O \sum_{i'=0}^{j} (s_j)_{i'} h_{i'} = 
  W_O \sum_{i'=0}^{j} \mathbb{I}(i' = i) h_{i'} = 
  W_O \; h_i
\]
