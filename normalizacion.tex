\documentclass{article}
\usepackage{graphicx} % Required for inserting images

\usepackage{amsmath}

\title{Normalización}
\author{GLyC}
\date{\today}

\begin{document}

\maketitle

Vamos a definir dos nociones de normalización de los transformers. Ambas van a hablar del valor del último vector antes de salir del ciclo. Lo que en el paper llamarían $h_n^L$. Además la matriz de output será \~ la identidad.

\section*{Normalización 0-1}

\subsection*{Definción}

Vamos a decir que un transformer $T_N$ es la 0-1 normalización de un transformer $T$ si el $h_n^L$ de $T_N$ cumple que:

\begin{equation*}
(h_n^L)_i =
\begin{cases}
0 & \text{si } i \neq 1 \\
x & \text{cc }
\end{cases}
\end{equation*}

donde 
\begin{align*}
    x = 0 & \text{ si $T$ responde no} \\
    x > 0 & \text{ si $T$ responde si}
\end{align*}






\section*{Normalización $\infty$}

\end{document}